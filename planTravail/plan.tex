\documentclass[10pt]{article}

\usepackage[utf8]{inputenc}
\usepackage[T1]{fontenc}
\usepackage[french]{babel}
\usepackage{textcomp}


\title{INFO-F-308 : Application Vivre ensemble - Plan de travail}
\author{Abdelkefi Amin \and Baudoux Nicolas \and Bauwin Lucie \and Boulif Ilias \and Dewit Maxime \and  Minhas Prabhdeep \and Requena Carlos \and Van Palm Geoffrey \and Zhou Mi}
\date{Année académique 2016-2017}


\begin{document}
 
 \maketitle
 
 \section{Calendrier d'avancement}
  \begin{tabular}{ l | c | r }
   Tâche & Equipe & Date \\ 
  \end{tabular}
 
 \section{Description des tâches}
  \subsection{Base de données}
   \subsubsection{Quelque chose}
   
  \subsection{Back-end général}
   \subsubsection{Algorithme de match}
    L’algorithme affichant les personnes à rencontrer se basera sur les informations fournies lors des questionnaires remplis précédemment
    par les utilisateurs. 
    Il établira un panel de 5 personnes ayant un pourcentage de différences plus ou moins grand.
    Nous définirons 5 intervalles de pourcentage (0\%-20\%, 21\%-40\%, 41\%-60\%, 61\%-80\% et 81\%-100\% de différences).   
    Dans le cas où l’algorithme est dans l’incapacité de trouver au moins une personne appartenant à un interval précis,
    il est prévu d’agrandir cet interval afin de faire correspondre une autre personne.
    Le choix des personnes se ferait aussi de manière à ne pas pouvoir retomber sur une personne déjà rencontrée. 
   \subsubsection{Limitation du nombre de choix}
    Par jour, l'utilisateur aura la possibilité de choisir parmi 5 fiches de personnes pour désigner la personne qu'il aimerait rencontrer. 
    Ces 5 fiches sont données par l'algorithmes de match présénté ci-dessus. 
    Celui-ci renvoie des personnes avec un pourcentage diffèrent pour permettre à l'utilisateur de choisir parmi des personnes ayant de 
    nombreux points en commun ou non.    

    Sur chaque fiche, l'utilisateur verra l'avatar de la personne accompagné de son surnom, de son niveau dans le jeu, ainsi que le 
    pourcentage de ressemblance avec celui-ci  
    Ce sont donc les informations disponibles sur la fiche qui permettront à l'utilisateur de choisir sa prochaine rencontre.

    Une fois la personne choisie, cette-dernière recevra une notification pour lui demander si elle accepte cette rencontr ou non.
   \subsubsection{Classement}
    Après avoir effectué une rencontre, l'utilisateur sera crédité d'un montant de points. Le nombre de points étant en lien avec le
    pourcentage de différence (plus de différences, plus de points. La somme des points donnera à l'utilisateur une position dans 
    un classement. L'accumulation de ces points donnera accès à des "niveaux.

    L'équipe Back-end implémentera donc différentes fonctions calculant les points marqués, actualisant le nombre total de points des
    utilisateurs et leur niveau.
   \subsubsection{Confirmation de rencontre}
    Afin de confirmer une rencontre entre deux utilisateurs, ceux-ci devront remplir un quiz supplémentaire. 
    Ce quiz a pour but de savoir si l'image qu'ils s'étaient fait l'un de l'autre est correcte. 

    Ensuite, ils reçoivent chacun un code et doivent se le partager pour valider la rencontre. 
    Une fois la rencontre validée, ils sont ajoutés dans leur liste d'amis respective où ils auront accès à une messagerie 
    instantanée et au profil de l'utilisateur rencontré afin de conserver un contact si la rencontre leur a plu. 
  \subsection{Front-end}
 
\end{document}
